%%
%% Test standard sentences for every fonts
%% 2004/02/06 Poonlap Veerathanabutr <poonlap@linux.thai.net>
%% 	      - rewrite this file
%%  
\documentclass[twocolumn,a4paper]{article}
\usepackage[english,thai]{babel}
\usepackage[utf8x]{inputenc}
\usepackage{fonts-arundina}


\begin{document}
\pagestyle{empty}
\vfil
\begin{figure*}
\Huge 
\hspace*{.3\textwidth}\usefont{LTH}{arunserif}{m}{n}แบบอักษรไทยใน {\latintext\LaTeX}\\
\hspace*{.3\textwidth}\usefont{LTH}{arunsans}{m}{n}แบบอักษรไทยใน {\latintext\LaTeX}\\
\hspace*{.3\textwidth}\usefont{LTH}{arunmono}{m}{n}แบบอักษรไทยใน {\latintext\LaTeX}\\
\end{figure*}
\vfil
\clearpage

\pagestyle{plain}
\section{\usefont{LTH}{arunserif}{b}{n}Arundina Serif}
\subsection{ตัวอย่างประโยคภาษาไทย\protect\footnote{โดยสมาคมคอมพิวเตอร์แห่งประเทศไทยในพระบรมราชูปถัมภ์}}

\usefont{LTH}{arunserif}{m}{n}
เป็นมนุษย์สุดประเสริฐเลิศคุณค่า\\
กว่าบรรดาฝูงสัตว์เดรัจฉาน \\
จงฝ่าฟันพัฒนาวิชาการ \\
อย่าล้างผลาญฤๅเข่นฆ่าบีฑาใคร\\ 
ไม่ถือโทษโกรธแช่งซัดฮึดฮัดด่า \\
หัดอภัยเหมือนกีฬาอัชฌาสัย \\
ปฏิบัติประพฤติกฎกำหนดใจ \\
พูดจาให้จ๊ะๆ จ๋าๆ น่าฟังเอยฯ\\ 

\usefont{LTH}{arunserif}{b}{n}
เป็นมนุษย์สุดประเสริฐเลิศคุณค่า\\
กว่าบรรดาฝูงสัตว์เดรัจฉาน \\
จงฝ่าฟันพัฒนาวิชาการ \\
อย่าล้างผลาญฤๅเข่นฆ่าบีฑาใคร\\ 
ไม่ถือโทษโกรธแช่งซัดฮึดฮัดด่า \\
หัดอภัยเหมือนกีฬาอัชฌาสัย \\
ปฏิบัติประพฤติกฎกำหนดใจ \\
พูดจาให้จ๊ะๆ จ๋าๆ น่าฟังเอยฯ\\ 

\usefont{LTH}{arunserif}{m}{it}
เป็นมนุษย์สุดประเสริฐเลิศคุณค่า\\
กว่าบรรดาฝูงสัตว์เดรัจฉาน \\
จงฝ่าฟันพัฒนาวิชาการ \\
อย่าล้างผลาญฤๅเข่นฆ่าบีฑาใคร\\ 
ไม่ถือโทษโกรธแช่งซัดฮึดฮัดด่า \\
หัดอภัยเหมือนกีฬาอัชฌาสัย \\
ปฏิบัติประพฤติกฎกำหนดใจ \\
พูดจาให้จ๊ะๆ จ๋าๆ น่าฟังเอยฯ\\ 

\usefont{LTH}{arunserif}{b}{it}
เป็นมนุษย์สุดประเสริฐเลิศคุณค่า\\
กว่าบรรดาฝูงสัตว์เดรัจฉาน \\
จงฝ่าฟันพัฒนาวิชาการ \\
อย่าล้างผลาญฤๅเข่นฆ่าบีฑาใคร\\ 
ไม่ถือโทษโกรธแช่งซัดฮึดฮัดด่า \\
หัดอภัยเหมือนกีฬาอัชฌาสัย \\
ปฏิบัติประพฤติกฎกำหนดใจ \\
พูดจาให้จ๊ะๆ จ๋าๆ น่าฟังเอยฯ\\ 

\subsection{ตัวอย่างภาษาอังกฤษ}
\usefont{LTH}{arunserif}{m}{n}
The quick brown fox jumps over the lazy dog.\\
\usefont{LTH}{arunserif}{b}{n}
The quick brown fox jumps over the lazy dog.\\
\usefont{LTH}{arunserif}{m}{sl}
The quick brown fox jumps over the lazy dog.\\
\usefont{LTH}{arunserif}{b}{sl}
The quick brown fox jumps over the lazy dog.\\
\usefont{LTH}{arunserif}{m}{it}
The quick brown fox jumps over the lazy dog.\\
\usefont{LTH}{arunserif}{b}{it}
The quick brown fox jumps over the lazy dog.\\
\usefont{LTH}{arunserif}{m}{n}
\MakeUppercase{The quick brown fox jumps over the lazy dog.}\\
\usefont{LTH}{arunserif}{b}{n}
\MakeUppercase{The quick brown fox jumps over the lazy dog.}\\
\usefont{LTH}{arunserif}{m}{sl}
\MakeUppercase{The quick brown fox jumps over the lazy dog.}\\
\usefont{LTH}{arunserif}{b}{sl}
\MakeUppercase{The quick brown fox jumps over the lazy dog.}\\
\usefont{LTH}{arunserif}{m}{it}
\MakeUppercase{The quick brown fox jumps over the lazy dog.}\\
\usefont{LTH}{arunserif}{b}{it}
\MakeUppercase{The quick brown fox jumps over the lazy dog.}\\



\subsection{การจัดระดับตัวอักษรและตัวอักษรพิเศษ}
\noindent
\usefont{LTH}{arunserif}{m}{n}
ที่ ท่า ทิ้ง ท้า กิ๊ง ก๊ง ตี๋ ต๋า บ่น ป่น, บ้น ป้น, บ๊น ป๊น, บ๋น ป๋น บิน ปิน บีน ปีน บิ่น ปิ่น บัน ปั่น บั่น ก็ ป็ ปู่ ญ ญุ ญู ญฺ ฐ ฐุ ฐู ฐฺ กุ ฎุ ฎู ฎฺ ฏุ ฏู ฏฺ บำ บ่ำ ปำ ป่ำ -\textyamakkan{} \textfongmun{} \textangkhankhu{} \textkhomut{} - -- --- `` '' ff fi tt ti AV\\
\usefont{LTH}{arunserif}{b}{n}
ที่ ท่า ทิ้ง ท้า กิ๊ง ก๊ง ตี๋ ต๋า บ่น ป่น, บ้น ป้น, บ๊น ป๊น, บ๋น ป๋น บิน ปิน บีน ปีน บิ่น ปิ่น บัน ปั่น บั่น ก็ ป็ ปู่ ญ ญุ ญู ญฺ ฐ ฐุ ฐู ฐฺ กุ ฎุ ฎู ฎฺ ฏุ ฏู ฏฺ บำ บ่ำ ปำ ป่ำ -\textyamakkan{} \textfongmun{} \textangkhankhu{} \textkhomut{} - -- --- `` '' ff fi tt ti AV\\
\usefont{LTH}{arunserif}{m}{it}
ที่ ท่า ทิ้ง ท้า กิ๊ง ก๊ง ตี๋ ต๋า บ่น ป่น, บ้น ป้น, บ๊น ป๊น, บ๋น ป๋น บิน ปิน บีน ปีน บิ่น ปิ่น บัน ปั่น บั่น ก็ ป็ ปู่ ญ ญุ ญู ญฺ ฐ ฐุ ฐู ฐฺ กุ ฎุ ฎู ฎฺ ฏุ ฏู ฏฺ บำ บ่ำ ปำ ป่ำ -\textyamakkan{} \textfongmun{} \textangkhankhu{} \textkhomut{} - -- --- `` '' ff fi tt ti AV\\
\usefont{LTH}{arunserif}{b}{it}
ที่ ท่า ทิ้ง ท้า กิ๊ง ก๊ง ตี๋ ต๋า บ่น ป่น, บ้น ป้น, บ๊น ป๊น, บ๋น ป๋น บิน ปิน บีน ปีน บิ่น ปิ่น บัน ปั่น บั่น ก็ ป็ ปู่ ญ ญุ ญู ญฺ ฐ ฐุ ฐู ฐฺ กุ ฎุ ฎู ฎฺ ฏุ ฏู ฏฺ บำ บ่ำ ปำ ป่ำ -\textyamakkan{} \textfongmun{} \textangkhankhu{} \textkhomut{} - -- --- `` '' ff fi tt ti AV\\

\subsection{ภาษาบาลี-สันสกฤต}
\usefont{LTH}{arunserif}{m}{n}
\textpali{หตฺเถสุ ภิกฺขเว สติ, อาทานนิกฺเขปนํ ปญฺญายติ}\\
\textpali{เอวเมว โข ภิกฺขเว}\\
\textpali{จกฺขุสมิํปิ สติ}\\
\textpali{จกฺขุสมฺผสฺสปจฺจยา อุปฺปชฺชติ อชฺฌตฺตํ สุขทุกขํ}\\
\textpali{ทิฏฺฐา มยา ภิกฺขเว ฉ ผสฺสายตนิกา นาม นิรยา}\\
\usefont{LTH}{arunserif}{b}{n}
\textpali{หตฺเถสุ ภิกฺขเว สติ, อาทานนิกฺเขปนํ ปญฺญายติ}\\
\textpali{เอวเมว โข ภิกฺขเว}\\
\textpali{จกฺขุสมิํปิ สติ}\\
\textpali{จกฺขุสมฺผสฺสปจฺจยา อุปฺปชฺชติ อชฺฌตฺตํ สุขทุกขํ}\\
\textpali{ทิฏฺฐา มยา ภิกฺขเว ฉ ผสฺสายตนิกา นาม นิรยา}\\
\usefont{LTH}{arunserif}{m}{it}
\textpali{หตฺเถสุ ภิกฺขเว สติ, อาทานนิกฺเขปนํ ปญฺญายติ}\\
\textpali{เอวเมว โข ภิกฺขเว}\\
\textpali{จกฺขุสมิํปิ สติ}\\
\textpali{จกฺขุสมฺผสฺสปจฺจยา อุปฺปชฺชติ อชฺฌตฺตํ สุขทุกขํ}\\
\textpali{ทิฏฺฐา มยา ภิกฺขเว ฉ ผสฺสายตนิกา นาม นิรยา}\\
\usefont{LTH}{arunserif}{b}{it}
\textpali{หตฺเถสุ ภิกฺขเว สติ, อาทานนิกฺเขปนํ ปญฺญายติ}\\
\textpali{เอวเมว โข ภิกฺขเว}\\
\textpali{จกฺขุสมิํปิ สติ}\\
\textpali{จกฺขุสมฺผสฺสปจฺจยา อุปฺปชฺชติ อชฺฌตฺตํ สุขทุกขํ}\\
\textpali{ทิฏฺฐา มยา ภิกฺขเว ฉ ผสฺสายตนิกา นาม นิรยา}\\
\vfil


\section{\usefont{LTH}{arunsans}{b}{n}Arundina Sans}
\subsection{ตัวอย่างประโยคภาษาไทย}

\usefont{LTH}{arunsans}{m}{n}
เป็นมนุษย์สุดประเสริฐเลิศคุณค่า\\
กว่าบรรดาฝูงสัตว์เดรัจฉาน \\
จงฝ่าฟันพัฒนาวิชาการ \\
อย่าล้างผลาญฤๅเข่นฆ่าบีฑาใคร\\ 
ไม่ถือโทษโกรธแช่งซัดฮึดฮัดด่า \\
หัดอภัยเหมือนกีฬาอัชฌาสัย \\
ปฏิบัติประพฤติกฎกำหนดใจ \\
พูดจาให้จ๊ะๆ จ๋าๆ น่าฟังเอยฯ\\ 

\usefont{LTH}{arunsans}{b}{n}
เป็นมนุษย์สุดประเสริฐเลิศคุณค่า\\
กว่าบรรดาฝูงสัตว์เดรัจฉาน \\
จงฝ่าฟันพัฒนาวิชาการ \\
อย่าล้างผลาญฤๅเข่นฆ่าบีฑาใคร\\ 
ไม่ถือโทษโกรธแช่งซัดฮึดฮัดด่า \\
หัดอภัยเหมือนกีฬาอัชฌาสัย \\
ปฏิบัติประพฤติกฎกำหนดใจ \\
พูดจาให้จ๊ะๆ จ๋าๆ น่าฟังเอยฯ\\ 

\usefont{LTH}{arunsans}{m}{it}
เป็นมนุษย์สุดประเสริฐเลิศคุณค่า\\
กว่าบรรดาฝูงสัตว์เดรัจฉาน \\
จงฝ่าฟันพัฒนาวิชาการ \\
อย่าล้างผลาญฤๅเข่นฆ่าบีฑาใคร\\ 
ไม่ถือโทษโกรธแช่งซัดฮึดฮัดด่า \\
หัดอภัยเหมือนกีฬาอัชฌาสัย \\
ปฏิบัติประพฤติกฎกำหนดใจ \\
พูดจาให้จ๊ะๆ จ๋าๆ น่าฟังเอยฯ\\ 

\usefont{LTH}{arunsans}{b}{it}
เป็นมนุษย์สุดประเสริฐเลิศคุณค่า\\
กว่าบรรดาฝูงสัตว์เดรัจฉาน \\
จงฝ่าฟันพัฒนาวิชาการ \\
อย่าล้างผลาญฤๅเข่นฆ่าบีฑาใคร\\ 
ไม่ถือโทษโกรธแช่งซัดฮึดฮัดด่า \\
หัดอภัยเหมือนกีฬาอัชฌาสัย \\
ปฏิบัติประพฤติกฎกำหนดใจ \\
พูดจาให้จ๊ะๆ จ๋าๆ น่าฟังเอยฯ\\ 

\subsection{ตัวอย่างภาษาอังกฤษ}
\usefont{LTH}{arunsans}{m}{n}
The quick brown fox jumps over the lazy dog.\\
\usefont{LTH}{arunsans}{b}{n}
The quick brown fox jumps over the lazy dog.\\
\usefont{LTH}{arunsans}{m}{it}
The quick brown fox jumps over the lazy dog.\\
\usefont{LTH}{arunsans}{b}{it}
The quick brown fox jumps over the lazy dog.\\
\usefont{LTH}{arunsans}{m}{n}
\MakeUppercase{The quick brown fox jumps over the lazy dog.}\\
\usefont{LTH}{arunsans}{b}{n}
\MakeUppercase{The quick brown fox jumps over the lazy dog.}\\
\usefont{LTH}{arunsans}{m}{it}
\MakeUppercase{The quick brown fox jumps over the lazy dog.}\\
\usefont{LTH}{arunsans}{b}{it}
\MakeUppercase{The quick brown fox jumps over the lazy dog.}\\



\subsection{การจัดระดับตัวอักษรและตัวอักษรพิเศษ}
\noindent
\usefont{LTH}{arunsans}{m}{n}
ที่ ท่า ทิ้ง ท้า กิ๊ง ก๊ง ตี๋ ต๋า บ่น ป่น, บ้น ป้น, บ๊น ป๊น, บ๋น ป๋น บิน ปิน บีน ปีน บิ่น ปิ่น บัน ปั่น บั่น ก็ ป็ ปู่ ญ ญุ ญู ญฺ ฐ ฐุ ฐู ฐฺ กุ ฎุ ฎู ฎฺ ฏุ ฏู ฏฺ บำ บ่ำ ปำ ป่ำ -\textyamakkan{} \textfongmun{} \textangkhankhu{} \textkhomut{} - -- --- `` '' ff fi tt ti AV\\
\usefont{LTH}{arunsans}{b}{n}
ที่ ท่า ทิ้ง ท้า กิ๊ง ก๊ง ตี๋ ต๋า บ่น ป่น, บ้น ป้น, บ๊น ป๊น, บ๋น ป๋น บิน ปิน บีน ปีน บิ่น ปิ่น บัน ปั่น บั่น ก็ ป็ ปู่ ญ ญุ ญู ญฺ ฐ ฐุ ฐู ฐฺ กุ ฎุ ฎู ฎฺ ฏุ ฏู ฏฺ บำ บ่ำ ปำ ป่ำ -\textyamakkan{} \textfongmun{} \textangkhankhu{} \textkhomut{} - -- --- `` '' ff fi tt ti AV\\
\usefont{LTH}{arunsans}{m}{it}
ที่ ท่า ทิ้ง ท้า กิ๊ง ก๊ง ตี๋ ต๋า บ่น ป่น, บ้น ป้น, บ๊น ป๊น, บ๋น ป๋น บิน ปิน บีน ปีน บิ่น ปิ่น บัน ปั่น บั่น ก็ ป็ ปู่ ญ ญุ ญู ญฺ ฐ ฐุ ฐู ฐฺ กุ ฎุ ฎู ฎฺ ฏุ ฏู ฏฺ บำ บ่ำ ปำ ป่ำ -\textyamakkan{} \textfongmun{} \textangkhankhu{} \textkhomut{} - -- --- `` '' ff fi tt ti AV\\
\usefont{LTH}{arunsans}{b}{it}
ที่ ท่า ทิ้ง ท้า กิ๊ง ก๊ง ตี๋ ต๋า บ่น ป่น, บ้น ป้น, บ๊น ป๊น, บ๋น ป๋น บิน ปิน บีน ปีน บิ่น ปิ่น บัน ปั่น บั่น ก็ ป็ ปู่ ญ ญุ ญู ญฺ ฐ ฐุ ฐู ฐฺ กุ ฎุ ฎู ฎฺ ฏุ ฏู ฏฺ บำ บ่ำ ปำ ป่ำ -\textyamakkan{} \textfongmun{} \textangkhankhu{} \textkhomut{} - -- --- `` '' ff fi tt ti AV\\

\subsection{ภาษาบาลี-สันสกฤต}
\usefont{LTH}{arunsans}{m}{n}
\textpali{หตฺเถสุ ภิกฺขเว สติ, อาทานนิกฺเขปนํ ปญฺญายติ}\\
\textpali{เอวเมว โข ภิกฺขเว}\\
\textpali{จกฺขุสมิํปิ สติ}\\
\textpali{จกฺขุสมฺผสฺสปจฺจยา อุปฺปชฺชติ อชฺฌตฺตํ สุขทุกขํ}\\
\textpali{ทิฏฺฐา มยา ภิกฺขเว ฉ ผสฺสายตนิกา นาม นิรยา}\\
\usefont{LTH}{arunsans}{b}{n}
\textpali{หตฺเถสุ ภิกฺขเว สติ, อาทานนิกฺเขปนํ ปญฺญายติ}\\
\textpali{เอวเมว โข ภิกฺขเว}\\
\textpali{จกฺขุสมิํปิ สติ}\\
\textpali{จกฺขุสมฺผสฺสปจฺจยา อุปฺปชฺชติ อชฺฌตฺตํ สุขทุกขํ}\\
\textpali{ทิฏฺฐา มยา ภิกฺขเว ฉ ผสฺสายตนิกา นาม นิรยา}\\
\usefont{LTH}{arunsans}{m}{it}
\textpali{หตฺเถสุ ภิกฺขเว สติ, อาทานนิกฺเขปนํ ปญฺญายติ}\\
\textpali{เอวเมว โข ภิกฺขเว}\\
\textpali{จกฺขุสมิํปิ สติ}\\
\textpali{จกฺขุสมฺผสฺสปจฺจยา อุปฺปชฺชติ อชฺฌตฺตํ สุขทุกขํ}\\
\textpali{ทิฏฺฐา มยา ภิกฺขเว ฉ ผสฺสายตนิกา นาม นิรยา}\\
\usefont{LTH}{arunsans}{b}{it}
\textpali{หตฺเถสุ ภิกฺขเว สติ, อาทานนิกฺเขปนํ ปญฺญายติ}\\
\textpali{เอวเมว โข ภิกฺขเว}\\
\textpali{จกฺขุสมิํปิ สติ}\\
\textpali{จกฺขุสมฺผสฺสปจฺจยา อุปฺปชฺชติ อชฺฌตฺตํ สุขทุกขํ}\\
\textpali{ทิฏฺฐา มยา ภิกฺขเว ฉ ผสฺสายตนิกา นาม นิรยา}\\
\vfil

\section{\usefont{LTH}{arunmono}{b}{n}Arundina Sans Mono}
\subsection{ตัวอย่างประโยคภาษาไทย}

\usefont{LTH}{arunmono}{m}{n}
เป็นมนุษย์สุดประเสริฐเลิศคุณค่า\\
กว่าบรรดาฝูงสัตว์เดรัจฉาน \\
จงฝ่าฟันพัฒนาวิชาการ \\
อย่าล้างผลาญฤๅเข่นฆ่าบีฑาใคร\\ 
ไม่ถือโทษโกรธแช่งซัดฮึดฮัดด่า \\
หัดอภัยเหมือนกีฬาอัชฌาสัย \\
ปฏิบัติประพฤติกฎกำหนดใจ \\
พูดจาให้จ๊ะๆ จ๋าๆ น่าฟังเอยฯ\\ 

\usefont{LTH}{arunmono}{b}{n}
เป็นมนุษย์สุดประเสริฐเลิศคุณค่า\\
กว่าบรรดาฝูงสัตว์เดรัจฉาน \\
จงฝ่าฟันพัฒนาวิชาการ \\
อย่าล้างผลาญฤๅเข่นฆ่าบีฑาใคร\\ 
ไม่ถือโทษโกรธแช่งซัดฮึดฮัดด่า \\
หัดอภัยเหมือนกีฬาอัชฌาสัย \\
ปฏิบัติประพฤติกฎกำหนดใจ \\
พูดจาให้จ๊ะๆ จ๋าๆ น่าฟังเอยฯ\\ 

\usefont{LTH}{arunmono}{m}{it}
เป็นมนุษย์สุดประเสริฐเลิศคุณค่า\\
กว่าบรรดาฝูงสัตว์เดรัจฉาน \\
จงฝ่าฟันพัฒนาวิชาการ \\
อย่าล้างผลาญฤๅเข่นฆ่าบีฑาใคร\\ 
ไม่ถือโทษโกรธแช่งซัดฮึดฮัดด่า \\
หัดอภัยเหมือนกีฬาอัชฌาสัย \\
ปฏิบัติประพฤติกฎกำหนดใจ \\
พูดจาให้จ๊ะๆ จ๋าๆ น่าฟังเอยฯ\\ 

\usefont{LTH}{arunmono}{b}{it}
เป็นมนุษย์สุดประเสริฐเลิศคุณค่า\\
กว่าบรรดาฝูงสัตว์เดรัจฉาน \\
จงฝ่าฟันพัฒนาวิชาการ \\
อย่าล้างผลาญฤๅเข่นฆ่าบีฑาใคร\\ 
ไม่ถือโทษโกรธแช่งซัดฮึดฮัดด่า \\
หัดอภัยเหมือนกีฬาอัชฌาสัย \\
ปฏิบัติประพฤติกฎกำหนดใจ \\
พูดจาให้จ๊ะๆ จ๋าๆ น่าฟังเอยฯ\\ 

\subsection{ตัวอย่างภาษาอังกฤษ}
\usefont{LTH}{arunmono}{m}{n}
The quick brown fox jumps over the lazy dog.\\
\usefont{LTH}{arunmono}{b}{n}
The quick brown fox jumps over the lazy dog.\\
\usefont{LTH}{arunmono}{m}{sl}
The quick brown fox jumps over the lazy dog.\\
\usefont{LTH}{arunmono}{b}{sl}
The quick brown fox jumps over the lazy dog.\\
\usefont{LTH}{arunmono}{m}{it}
The quick brown fox jumps over the lazy dog.\\
\usefont{LTH}{arunmono}{b}{it}
The quick brown fox jumps over the lazy dog.\\
\usefont{LTH}{arunmono}{m}{n}
\MakeUppercase{The quick brown fox jumps over the lazy dog.}\\
\usefont{LTH}{arunmono}{b}{n}
\MakeUppercase{The quick brown fox jumps over the lazy dog.}\\
\usefont{LTH}{arunmono}{m}{sl}
\MakeUppercase{The quick brown fox jumps over the lazy dog.}\\
\usefont{LTH}{arunmono}{b}{sl}
\MakeUppercase{The quick brown fox jumps over the lazy dog.}\\
\usefont{LTH}{arunmono}{m}{it}
\MakeUppercase{The quick brown fox jumps over the lazy dog.}\\
\usefont{LTH}{arunmono}{b}{it}
\MakeUppercase{The quick brown fox jumps over the lazy dog.}\\



\subsection{การจัดระดับตัวอักษรและตัวอักษรพิเศษ}
\noindent
\usefont{LTH}{arunmono}{m}{n}
ที่ ท่า ทิ้ง ท้า กิ๊ง ก๊ง ตี๋ ต๋า บ่น ป่น, บ้น ป้น, บ๊น ป๊น, บ๋น ป๋น บิน ปิน บีน ปีน บิ่น ปิ่น บัน ปั่น บั่น ก็ ป็ ปู่ ญ ญุ ญู ญฺ ฐ ฐุ ฐู ฐฺ กุ ฎุ ฎู ฎฺ ฏุ ฏู ฏฺ บำ บ่ำ ปำ ป่ำ -\textyamakkan{} \textfongmun{} \textangkhankhu{} \textkhomut{} - -- --- `` '' ff fi tt ti AV\\
\usefont{LTH}{arunmono}{b}{n}
ที่ ท่า ทิ้ง ท้า กิ๊ง ก๊ง ตี๋ ต๋า บ่น ป่น, บ้น ป้น, บ๊น ป๊น, บ๋น ป๋น บิน ปิน บีน ปีน บิ่น ปิ่น บัน ปั่น บั่น ก็ ป็ ปู่ ญ ญุ ญู ญฺ ฐ ฐุ ฐู ฐฺ กุ ฎุ ฎู ฎฺ ฏุ ฏู ฏฺ บำ บ่ำ ปำ ป่ำ -\textyamakkan{} \textfongmun{} \textangkhankhu{} \textkhomut{} - -- --- `` '' ff fi tt ti AV\\
\usefont{LTH}{arunmono}{m}{it}
ที่ ท่า ทิ้ง ท้า กิ๊ง ก๊ง ตี๋ ต๋า บ่น ป่น, บ้น ป้น, บ๊น ป๊น, บ๋น ป๋น บิน ปิน บีน ปีน บิ่น ปิ่น บัน ปั่น บั่น ก็ ป็ ปู่ ญ ญุ ญู ญฺ ฐ ฐุ ฐู ฐฺ กุ ฎุ ฎู ฎฺ ฏุ ฏู ฏฺ บำ บ่ำ ปำ ป่ำ -\textyamakkan{} \textfongmun{} \textangkhankhu{} \textkhomut{} - -- --- `` '' ff fi tt ti AV\\
\usefont{LTH}{arunmono}{b}{it}
ที่ ท่า ทิ้ง ท้า กิ๊ง ก๊ง ตี๋ ต๋า บ่น ป่น, บ้น ป้น, บ๊น ป๊น, บ๋น ป๋น บิน ปิน บีน ปีน บิ่น ปิ่น บัน ปั่น บั่น ก็ ป็ ปู่ ญ ญุ ญู ญฺ ฐ ฐุ ฐู ฐฺ กุ ฎุ ฎู ฎฺ ฏุ ฏู ฏฺ บำ บ่ำ ปำ ป่ำ -\textyamakkan{} \textfongmun{} \textangkhankhu{} \textkhomut{} - -- --- `` '' ff fi tt ti AV\\

\subsection{ภาษาบาลี-สันสกฤต}
\usefont{LTH}{arunmono}{m}{n}
\textpali{หตฺเถสุ ภิกฺขเว สติ, อาทานนิกฺเขปนํ ปญฺญายติ}\\
\textpali{เอวเมว โข ภิกฺขเว}\\
\textpali{จกฺขุสมิํปิ สติ}\\
\textpali{จกฺขุสมฺผสฺสปจฺจยา อุปฺปชฺชติ อชฺฌตฺตํ สุขทุกขํ}\\
\textpali{ทิฏฺฐา มยา ภิกฺขเว ฉ ผสฺสายตนิกา นาม นิรยา}\\
\usefont{LTH}{arunmono}{b}{n}
\textpali{หตฺเถสุ ภิกฺขเว สติ, อาทานนิกฺเขปนํ ปญฺญายติ}\\
\textpali{เอวเมว โข ภิกฺขเว}\\
\textpali{จกฺขุสมิํปิ สติ}\\
\textpali{จกฺขุสมฺผสฺสปจฺจยา อุปฺปชฺชติ อชฺฌตฺตํ สุขทุกขํ}\\
\textpali{ทิฏฺฐา มยา ภิกฺขเว ฉ ผสฺสายตนิกา นาม นิรยา}\\
\usefont{LTH}{arunmono}{m}{it}
\textpali{หตฺเถสุ ภิกฺขเว สติ, อาทานนิกฺเขปนํ ปญฺญายติ}\\
\textpali{เอวเมว โข ภิกฺขเว}\\
\textpali{จกฺขุสมิํปิ สติ}\\
\textpali{จกฺขุสมฺผสฺสปจฺจยา อุปฺปชฺชติ อชฺฌตฺตํ สุขทุกขํ}\\
\textpali{ทิฏฺฐา มยา ภิกฺขเว ฉ ผสฺสายตนิกา นาม นิรยา}\\
\usefont{LTH}{arunmono}{b}{it}
\textpali{หตฺเถสุ ภิกฺขเว สติ, อาทานนิกฺเขปนํ ปญฺญายติ}\\
\textpali{เอวเมว โข ภิกฺขเว}\\
\textpali{จกฺขุสมิํปิ สติ}\\
\textpali{จกฺขุสมฺผสฺสปจฺจยา อุปฺปชฺชติ อชฺฌตฺตํ สุขทุกขํ}\\
\textpali{ทิฏฺฐา มยา ภิกฺขเว ฉ ผสฺสายตนิกา นาม นิรยา}\\
\vfil


\end{document}
